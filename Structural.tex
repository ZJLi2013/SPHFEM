%========================================================================
\documentclass[11pt]{article}
\usepackage{graphics}
\graphicspath{{figs/}}

%===========================================
% Fix the margins
\setlength{\topmargin}{-0.4in}    % distance from top of page to begining of text
\setlength{\textheight}{9.0in}    % height of main text
\setlength{\textwidth}{6.5in}     % width of text
\setlength{\oddsidemargin}{0.in}  % odd page left margin
\setlength{\evensidemargin}{0.in} % even page left margin

%===========================================
\usepackage{amssymb,amsfonts}
\usepackage{amsmath}

%===========================================
\begin{document}

\title{Structural Equations}

\author{Zhengjiang Li}

\date{}

\maketitle

\section{Beam vs. Plane}
Beam should be good, while the reference has a simulation on plat plane, so I follow this work.

\section{Strong Form to Matrix Form}

strong form of dynamic/vibration can be descripted as:
  
  \[
  \begin{cases}
  $$ \sigma_{ji,j} + b_i = \rho \ddot{u_i} on \Omega $$ \\
  $$ u_i = 0  on \Gamma with fixed displacement $$ \\
  $$ \sigma_ij \cdot n_j = p_i on \Gamma with moving interface $$ \\
  $$ u_i(t=0) = u_0$$ \\
  $$ \dot{u_i(t=0)} = \dot{u_{i0}} $$ 
  \end{cases} 
  \]

Note,in fluid structural interaction applications, obviously there are two kind of boundaries. the fixed displacement B.C. and the fraction(fluid pressure) B.C.

Galerkin weak form obtained as:

  $$ \delta \Pi =  \int_{\Omega} (\delta u_i \rho \ddot{u_i} + \delta \varepsilon_{ij} \sigma_{ij} - \delta u_i b_i )dv  - \int_{\Gamma} \delta u_i t_i ds = 0 $$

 in which, $t_i$ is boundary traction.

 
consider linear elasticity relationship $ \sigma = C \varepsilon $
, strain-displacement relationship $ \varepsilon = B \hat{u}$ , and shape function $N$ to obtain displacement $u(\vec{x}, t) = N u_i$.

 $$ \delta \Pi = \sum {\delta \Pi}^e = \sum \delta {u^e}^T (\int_{\Omega^e} N^T \rho N dv \ddot{u^e} + \int_{\Omega^e} B^T C B dv {u^e} - \int_{\Omega^e} N^T b dv - \int_{\Gamma^e} N^T t ds ) = 0 $$

from above, we obtain equation for an element$ M^e \ddot{u^e} + K^e {u^e} = f^e $

where $$ M^e = \int N^t \rho N dv $$

	$$ K^e = \int B^T C B dv $$

	$$ f^e = \int N^T b dv + N^T t ds $$

\section {plate stress element}
in 2D simulation, we choice the 4-node plate stress element, which has two D.o.F. at each node. (Note, a plate element will have 3 D.O.F at each node, w displacement, $\theta_x$ $\theta_y$. For Kirchoff thin plate/shell theory, $\theta_x$ $\theta_y$ is derivative of w in terms of x and y respectively; and a shell element is the combination of a plate stress element(used to describe in-plane membrane stress) and a plate element(used to describe out-plane bending stress)) 

with bilinear shape functions:

$$N_1 = \frac{(1-\xi)(1-\eta)}{4}$$
$$N_2 = \frac{(1+\xi)(1-\eta)}{4}$$
$$N_3 = \frac{(1+\xi)(1+\eta)}{4}$$
$$N_4 = \frac{(1-\xi)(1+\eta)}{4}$$

the derivative of a scalar or a vector component in terms of the physical coordinate can be mapped to the same corresponding derivative in $\xi-\eta$ local coordinate by Jacobian Transformation. 



\section {traction boundary}

In this couple algorithm,interface transfer happens as following:

\begin{enumerate}
\item 
  at the first first step, the interface boundary is the initial configuration/profile of the structure
\item
  the interface boundary act like repulsive particles, and which offer the extern force in SPH solver. call SPH solver and  update fluid particles information.
\item
  to keep momemtum conservation, add opposite repulsive force on structure boundary as the traction B.C. in FEM solver. call FEM solver, update the boundary position and return to step one. 
\end{enumerate}




\end{document}
