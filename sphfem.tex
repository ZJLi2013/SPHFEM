%========================================================================
\documentclass[11pt]{article}
\usepackage{graphics}
\graphicspath{{figs/}}

%===========================================
% Fix the margins
\setlength{\topmargin}{-0.4in}    % distance from top of page to begining of text
\setlength{\textheight}{9.0in}    % height of main text
\setlength{\textwidth}{6.5in}     % width of text
\setlength{\oddsidemargin}{0.in}  % odd page left margin
\setlength{\evensidemargin}{0.in} % even page left margin

%===========================================
\usepackage{amssymb,amsfonts}
\usepackage{amsmath}

%===========================================
\begin{document}

\title{2D fluid beam couple by SPH and FEM}

\author{Zhengjiang Li}

\date{}

\maketitle



\section {system governing equation}

In this couple system, the self-dependent variable are particles' velocity vector $v(x,y,t)$ and FEM node position vector $w(x,t)$. so system governing equations are:

   $$	\rho  \dot{v} = \nabla \sigma + g + f^{s2f} $$ 
	
while as considering geometric nonlienear beam, the balance equation is not that explicit, what we can write down here is the energy-based equation:

$$\int (S_{xx} \delta E_{xx}^T) dx =  \int (-\rho \delta \ddot{w} + q \delta w ) dx $$
  
where q at RHS is the average traction of nodal force from fluid to beam $f^{f2s}$.
\begin{subequations}
\begin{align}
 \int_0^L q(x,t) dx = \sum_{i=1}^{np} f^{f2s}_i \\
q_i = \frac{f_i + f_{i+1}}{2l_e}, \  1< i < np \\
q_1 = (f_1 + f_2/2)/ l_e \\
q_{np} = (f_{np-1}/2 + f_{np})/l_e 
\end{align}
\end{subequations}

where $ q(x,t)$ is the equivalent distributed traction, and $np$ is the number of interface nodes. $q_i$ stands for a constant traction on an element length. 


The two equations are coupled by $f^{s2f}$ and $f^{f2s}$. as one FEM node correspond to multi fluid particles, while each fluid particles correspond to at most 2 FEM node, so these two forces are actually not equivalent.


Note in this simple case, all beam nodes play dual-roles as FEM node as well as interface nodes. One way is to set interface node - fluid particle Pair $<i, I>$, here $i$ stands for fluid particle, and $I$ stands for interface node, which obtain:

$$ f^{s2f} = \sum_i repulse\_force(<i,I>) $$
$$ f^{f2s} = \sum_I -repulse\_force(<i,I>) $$

\section{weak form of structural governing equation}

\begin{enumerate}

\item{kinematic equation}

An accurate 2D kinemtic relationship is, details in \cite[p.~215]{limingrui}

\begin{subequations}
 \begin{align}
	U &= u - z \sin \theta \\
	W &= w - z(1- \cos \theta)
  \end{align}
 \end{subequations}

here, $u, w$ are displacements of neutral axis, and $\theta$ is rotational angle of an arbitrary cross-section.

\item{displacemnt-strain equation}

taking derivativs of kinematic equation, in terms of $x,y$ respectively:

 \begin{subequations}
	\begin{align}
	U_x = u_x - z \theta_x \cos \theta \\
	U_z = - \sin \theta \\
	W_x = w_x - z \theta_x \sin \theta \\
	W_z = -(1- \cos \theta)
	\end{align}
\end{subequations}

Giving Green-Lagrangian strain:
 \begin{subequations}
	\begin{align}
	E_{xx} = U_x + ( U_x^2 + W_x^2)/2 = u_x - z \theta_x \cos \theta = u_x + ( u_x^2 + w_x^2)/2 + z^2 \theta_x^2/2 - z \theta_x R_{es}  \\
	E_{xz} = U_x + W_x + U_x U_z + W_x W_z = R_{en} \\ 
	E_{zz} = W_z + ( U_z^2 + W_z^2)/2 = 0
	\end{align}
\end{subequations}

in which, $ R_{es} = (1 + u_x) \cos \theta + v_x \sin \theta $, and $R_{en} = v_x \cos \theta - ( 1 + u_x) \sin \theta$

and define $ \epsilon =  u_x + ( u_x^2 + w_x^2)/2 $ 


\item {virtual work } \\
In finite deformation theory, Green-Lagrangian Strain $E_{xx}$ is energy-conjugate couple correspond to Second Piola-Kirchhoff stress $S_{xx}$. so the internal force virtual work is, taking into the strain defined above

$$ \delta W^{in} = \int (S_{xx} \delta E_{xx}^T) dV =  \int ( S_{xx} \delta E_{xx})dV + \int (S_{xz} \delta E_{xz} )dV $$

ignoring item $y^2(\theta_x)^2$, obtain the following: 
\begin{subequations}
	\begin{align}
	\int S_{xx} \delta E_{xx} dV = \int (\int S_{xx} dA) \delta \epsilon dx + \int (\int-y S_{xx}dA) \delta(\theta_x R_{es}) dx \\
	\int S_{xz} \delta E_{xz} dV = \int (\int S_{xz} dA) \delta R_{en} dx
	\end{align}
\end{subequations}

Define section Kirchoff internal force(also called, generalized stress) as following:

\begin{subequations}
	\begin{align}
	N = \int S_{xx} dA
	Q = \int S_{xz} dA
	M = \int -y S_{xx} dA
	\end{align}
\end{subequations}

so the internal Kirchoff stress virtual work can be expressed again as:

$$ \delta W^{in} = \int_L(N \delta \epsilon + Q \delta R_{en} + M \delta (\theta_x R_{es}) ) dx $$

From this equation compared, we can see that in the large-rotation case(one kind of geometric nonlinear), normal stress lead to both bending moment and axial force, which is always ignored under small deformation assumption.

For approximation application, we can even derive the balance equation based on displacement from here:
$$ \rho \frac{\partial^2 w}{\partial t^2} + \frac{\partial^2}{\partial x^2}( EI \frac{\partial^2 w}{\partial x^2}) - N(x,t) \frac{\partial^2 w}{\partial x^2} = q(x,t) $$

in this paper, we still keep on the original equation directly from virtual work. 

consider traction force and D'Alembert force , the out force virtual work is(here we ignore the axial vibration):
$$ \delta W^{out} = \int -\rho \ddot{\delta w} + q \delta w dx $$


\item{weak form}

Before we keep on, we need to take a look at infinite length change ratio $r$ during deformation. Let $ds^0$ be infinite length at $t=0$, and $ds^t$ be the length at time $t$. so for 1D beam,
	$$ ds^0 = dx $$
	$$ (ds^t)^2 = (dx + du)^2 + dw^2 $$
which gives $r = ((dx + du)^2 + dw^2)^(1/2)$, which actually is the length of netural axis at time $t$.Define $\varphi$ as the angle between netural axis and y-direction at time $t$. To make it simple, another assumption here is the neutral axial deformation in axil direction is small, so $|r|=1$ is proper.

Define $\gamma = \varphi - \theta$, which is the shear angle, and $\theta$ is the cross-section rotational angle.

so $R_{es} = cos(\gamma)$ and $R_{en} = sin(\gamma)$. 

Now let's derive the virtual generalized strain correspond to each generalized Kirchoff stress:

\begin{subequations}
\begin{align}
  \delta \epsilon = (1+ u_x) \delta u_x + w_x \delta w_x = \cos \varphi \delta u_x + \sin \varphi \delta w_x \\
 \delta R_{en} \approx \delta \sin(\varphi - \theta) = \delta (w_x \cos \theta - (1+u_x) \sin \theta) = \delta w_x \cos \theta - \delta u_x \sin \theta - \cos \gamma \delta \theta \\ 
  \delta (\theta _x R_{es}) \approx  \delta \theta_x 
\end{align}
\end{subequations}


In our SPH-FEM system, we don't even care about axial displacement. so a much simpler virtual generalized strain, which ignore $u$, can be expressed as:
\begin{subequations}
\begin{align}
  \delta \epsilon \approx \sin \theta \delta w_x \\
 \delta R_{en} \approx \delta w_x \cos \theta - \cos \gamma \delta \theta \\ 
 \delta (\theta_x R_{es}) \approx = \delta \theta_x 
\end{align}
\end{subequations}

\textbf{does this make sense??}
 
\item{Newton-Raphson Linearization} 

consider linear elastic strain-stress relationship, the virtual work can be written in bracket form:

$$ \Pi(u,F) = [D \varepsilon; \delta \varepsilon] - [f; \delta u] $$

in which, Virtual Energy $\Pi$ is the function of displacement and its gradient. 

At $t_{n+1}$, using Tayler Expansion, obtain:

$$\Pi(u,F)^{t_{n+1}} = \Pi(u,F)^{t_n} + \frac {\partial \Pi}{\partial u} (u_{n+1} - u_n)  + \frac {\partial \Pi}{\partial F} ( g_{n+1} - g_n) = 0 $$

In the usual  writting way:

$$\frac {\partial \Pi}{\partial u} u_{n+1}  + \frac {\partial \Pi}{\partial F} g_{n+1} = \frac {\partial \Pi}{\partial u} u_{n+1}  + \frac {\partial \Pi}{\partial F} g_{n+1} - Pi(u,F)^{t_n} $$

in which, 
$$ \frac{\partial \Pi}{\partial u} \delta u +   \frac {\partial \Pi}{\partial F} \delta  g = [D \frac{\partial \varepsilon}{\partial u}; \delta \varepsilon] + [D \varepsilon; \frac{\partial \delta \varepsilon}{\partial u}] $$

as $\frac{\partial[f;u]}{\partial u} = 0 $

From above, the first part of RHS is symmetric, while the second part of RHS is non-symmetric. To construct robost tangent stiffness matrix, we can ignore the non-symmetric part, which works like an pre-conditioner. 
 
$$[D \frac{\partial \varepsilon}{\partial u}; \delta \varepsilon]^{n+1}= ( [D \frac{\partial \varepsilon}{\partial u}; \delta \varepsilon] + [D \varepsilon; \delta \varepsilon)^n - ( [D \varepsilon; \delta \varepsilon] - [f; \delta u])^n $$


From section 5(weak form), now we can write down the final linearized weak form as:

\begin{equation}
\begin{split}
[ EA( \cos \varphi \frac{du}{dx} + \sin \varphi {dv}{dx}); ( \cos \varphi \frac{d \delta u}{dx} + \sin \varphi {d \delta v}{dx})] + [EI \frac{d \theta}{dx};\frac{d \delta \theta}{dx}] \\
 + [ GA ( \cos \theta \frac{dv}{dx} - \sin \theta \frac{du}{dx} - \cos \gamma \theta); ( \cos \theta \frac{d \theta v}{dx} - \sin \theta \frac{d \theta u}{dx} - \cos \gamma \theta \theta)] \\
 = [f; \delta] 
\end{split}
\end{equation}

\item{matrix approximation}

virtual work equation above is easy to implement based on Timoshenko Beam Assumption; and weak form from balance equation is based on Euler-Boulluni Beam Assumption. so we can use either linear interpolations for $u, w, \theta$ as TB or linear interpolation for $u$ and Hermite interpolation for $w$ as EB.

For large rotation, Timonshenko Theory is better, and it's easy to implement on both Total Langrange Method and Update Langrange Method. While Euler-Boulluni Theory is  an good approximate, but strict to UL implementation.

in the following, we will introduce these two ways.

\subitem{ EU Beam element} 

EU Beam assumption that $\theta = w_x$, so for $w$ taked Hermite Interpolation:
$$ w(x,t) = w_1(t) \phi_1(x) + \theta_1(t) \phi_2(x) + w_2(t) \phi_3(x) + \theta_2(t) \phi_4(x) $$

Define master element on length $l_e$ with coordinate $\xi$, ranging from $[-1, 1]$

$$ \phi_1(x) = \frac{1}{4} (1-\xi)^2 (2+\xi) $$
$$ \phi_2(x) = \frac{l_e}{8}(1-\xi)^2(1+\xi) $$
$$ \phi_3(x) = \frac{1}{4} (1+\xi)^2 (2-\xi) $$
$$ \phi_4(x) = \frac{l_e}{8}(1+\xi)^2(\xi-1) $$

define $[w_1, \theta_1, w_2, \theta_2] = [ \Delta_1, \Delta_2,\Delta_3, \Delta_4] $

from master element to physical element we need define an Jacobian Transformation, which is simple for 2-node  beam element  as $ x = \frac{l_e}{2} (\xi + 1)$
$$ J = \frac{\partial \xi}{\partial x} = 2/l_e $$


in an element, the following satisfy:
$$ \sum_{i=1}^4 \sum_{j=1}^4 ( \int_{-1}^{1}( EI \frac{d^2 \phi_i}{dx^2} \frac{d^2 \phi_j}{dx^2}\Delta_j + \rho \frac{d^2 \Delta_j}{dt^2} \phi_i \phi_j + N^L \frac{d \phi_i}{dx} \frac{d \phi_j}{dx} \Delta_j) dx) = \int_{-1^{1}} \phi_i q(x,t) dx $$  

From local coordinate system to global coordiante system, we need define transformation matrix $T$

$$
\{ \begin{array}{c} w^l \\ \theta^l \end{array} \} = \begin{bmatrix} \cos \varphi & 0 \\ 0 & 1 \end{bmatrix} \{ \begin{array}{c} w^g \\ \theta^g \end{array} \}
$$

Define element stiff matrix, element mass matrix and element force as :

\begin{subequations}
\begin{align}
k^e = \sum_{i=1}^4 \sum_{j=1}^4 ( \int_{x_i}^{x_{i+1}}( EI \frac{d^2 \phi_i}{dx^2} \frac{d^2 \phi_j}{dx^2}\Delta_j + N^L \frac{d \phi_i}{dx} \frac{d \phi_j}{dx} \Delta_j) dx) \\
m^e =  \sum_{i=1}^4 \sum_{j=1}^4 ( \int_{x_i}^{x_{i+1}} \rho \frac{d^2 \Delta_j}{dt^2} \phi_i \phi_j dx \\
f^e =  \sum_{i=1}^4 ( \int_{x_i}^{x_{i+1}} \int_{x_i}^{x_{i+1}} \phi_i q(x,t) dx
\end{align}
\end{subequations}

Assembly up, we obtain the whole structural system Matrix Format:

$$ \sum_{ne} T^T m^e T \{ \ddot{\Delta} \} + \sum_{ne} T^T k^e T \{ \Delta \} = \sum_{ne} T^T f^e $$ 

where $ne$ is the total number of elements, $T$ is the transformation matrix.

Define global Mass Matrix, Stiff Matrix and Force Vector as

$$ M = \sum_{ne} T^T m^e T $$
$$ K = \sum_{ne} T^T k^e T $$
$$ F = \sum_{ne} T^T f^e $$ 

\textbf{EU will not be used in this project}

\subitem{Timoshenko beam element}

From section 6, it is clear to see there should be three variables in each node( namely 3 DOF) $u, w, \theta$. 
and Timoshenko, who didn't assumpte $\theta = w_x$,  works better for large rotation/deformation case.
Another consideration of TBE is element accuracy, since we have displacement gradient in the weak form, so it's better to use at least 2-order element with at least 3 nodes per element. 

element looks like:

 |------|------|
 1(-1)  3(0)   2(1)

So the shape function now

$$ phi_1 = \xi (\xi -1 )/2 $$
$$ phi_2 = \xi (\xi + 1)/2 $$
$$ phi_3 = (1 - \xi ^2) $$

 

the element mass matrix and element force vector are same as EB beam element, and matrix assembly is same as EB. 
   
\end{enumerate}

\section{SPH fluid equation}

assuming weak-compressible, the mass conservation equation is

$$ \frac{d \rho}{dt} = - \rho \nabla  v $$

momentum equation is 

$$ \frac{dv}{dt} = - \frac{1}{\rho} \nabla P + \Gamma + g $$

where $v$ is fluid particle velocity, and $\Gamma$ refers to dissipative terms and $ g=(0,0,-9.81) m/s^{-2} $. Basically two ways to implement:

\begin{enumerate}
\item{artificial viscosity}

artificial viscosity proposed by Monaghan(1992), the equation above is:

$$  \frac{dv_i}{dt} = - \sum_{j} m_j (\frac{P_i}{\rho _i ^2} + \frac{P_j}{\rho_j ^2} + \Pi _{ij} ) \nabla_i W_{ij} + g $$

where viscosity term $\Pi_{ij}$ is given by

$$ \Pi_{ij} = \{ \begin{array}{c} \frac{-\alpha \bar{c_{ij}} \mu_{ij}}{\rho _{ij}} \; v_{ij} \cdot r_{ij} < 0 \\ 0 \;  v_{ij} \cdot r_{ij} > 0 \end{array} $$

where $ \mu_{ij} = \frac{h v_{ij} r_{ij}}{ r_{ij}^2 + \eta ^2} $, $\eta^2 = 0.01 h^2$, $\bar{c_{ij}} = \frac{1}{2}(c_i+c_j)$ is the mean sound speed. $\alpha = 0.3$, $c=10$  for DamBreak case. 

\item{laminar viscosity}
proposed by Lo and Shao (2002)$$  \frac{dv_i}{dt} = - \sum_{j} m_j (\frac{P_i}{\rho _i ^2} + \frac{P_j}{\rho_j ^2}) \nabla_i W_{ij} + g + \sum_{j} \frac{4 \nu_0 r_{ij} \cdot \nabla_i w_{ij}}{(\rho_i + \rho_j)(r_{ij}^2+\eta^2)}v_{ij} $$
$$  \frac{dv_i}{dt} = - \sum_{j} m_j (\frac{P_i}{\rho _i ^2} + \frac{P_j}{\rho_j ^2}) \nabla_i W_{ij} + g + \sum_{j} \frac{4 \nu_0 r_{ij} \cdot \nabla_i w_{ij}}{(\rho_i + \rho_j)(r_{ij}^2+\eta^2)}v_{ij} $$
where $\nu_0$ is kinetic viscosity $ 10^{-6} m^2s^{-1} $

\end{enumerate}

choose Quadratic Kernel function

$$w(r_{ij},h)  = \alpha_D [ \frac{3}{16} q^2 - \frac{3}{4} q + \frac{3}{4}] \; 0 \leq q \leq 2 $$

the frist gradient is $ \nabla_i w_{ij} = \alpha_D (\frac{3}{8} q - \frac{3}{4})$

where $ q= r_{ij}/h $, $alpha_D = \frac{2}{\pi h^2}$ for 2D case.

For artificial viscosity, define

$$ L = \sum_j \frac{m_j \alpha c_{ij}}{\rho _ {ij}} \frac{h r_{ij}}{r_{ij}^2 + \eta^2} \nabla_i w_{ij}$$ when $ v_{ij} \cdot r_{ij} < 0 $ else $ L =0 ,  v_{ij} \cdot r_{ij} > 0$

For laminar viscosity, define
$$ L = \sum_{j} \frac{4 \nu_0 r_{ij} \cdot \nabla_i w_{ij}}{(\rho_i + \rho_j)(r_{ij}^2+\eta^2)} $$ 

and for both artificial viscosity and laminar viscosity, we have the term unrelated to velocity, defined as

$$ RHS_0 =  - \sum_{j} m_j (\frac{P_i}{\rho _i ^2} + \frac{P_j}{\rho_j ^2}) \nabla_i W_{ij} + g $$

so sph governing equation in matrix format is 
$ \ddot{v_i} - L v_{ij} + RHS_0 $


To update density of fluid particles, we use 

$$ \frac{d \rho_i}{dt} = \sum_j m_j v_{ij} \nabla_i w_{ij} $$

and to update pressure we use Monaghan(1994) state equation as

$$ P = B[ (\frac{\rho}{\rho_0})^ \kappa - 1] $$

where $\kappa = 7$, $B= c_0^2 \rho_0 /\kappa$, $\rho_0 = 1000 kgm^{-3}$ 

we can see here, the update of particles density and pressure are not directly related to the coupled system. so these two parameters are not solve simultaneous with particle velocities $v$ in final.


\section{coupled system approximation}

here use EB beam element
 $$ \begin{bmatrix} M && 0 \\ 0 && I \end{bmatrix} \{ \begin{array}{c} \ddot{\Delta_J} \\ \dot{v} \end{array} \} + \begin{bmatrix} K && 0 \\ 0 && L \end{bmatrix} \{ \begin{array}{c} \Delta_J \\ v \end{array} \} = \{ \begin{array}{c} f^{f2s}\\ f^{s2f}+RHS_0 \end{array} \} $$

where $ \Delta_J$ stands for displacement in structural, $v$ stands for velocity of particles. 

\section{Reference}
% \bibliography


\end{document}
