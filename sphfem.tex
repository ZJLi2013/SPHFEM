%========================================================================
\documentclass[11pt]{article}
\usepackage{graphics}
\graphicspath{{figs/}}

%===========================================
% Fix the margins
\setlength{\topmargin}{-0.4in}    % distance from top of page to begining of text
\setlength{\textheight}{9.0in}    % height of main text
\setlength{\textwidth}{6.5in}     % width of text
\setlength{\oddsidemargin}{0.in}  % odd page left margin
\setlength{\evensidemargin}{0.in} % even page left margin

%===========================================
\usepackage{amssymb,amsfonts}
\usepackage{amsmath}

%===========================================
\begin{document}

\title{2D fluid beam couple by SPH and FEM}

\author{Zhengjiang Li}

\date{}

\maketitle



\section {system governing equation}

In this couple system, the self-dependent variable are particles' velocity vector $v(x,y,t)$ and FEM node position vector $w(x,t)$. so system governing equations are:

   $$	\rho  \dot{v} = \nabla \sigma + g + f^{s2f} $$ 
   $$	\frac{d^2}{dx^2}(EI \frac{d^2w}{dx^2} ) + \rho \ddot{w} = f^{f2s} $$

The two equations are coupled by $f^{s2f}$ and $f^{f2s}$. as one FEM node correspond to multi fluid particles, while each fluid particles correspond to at most 2 FEM node, so these two forces are actually not equivalent.


Note in this simple case, all beam nodes play dual-roles as FEM node as well as interface nodes. One way is to set interface node - fluid particle Pair $<i, I>$, here $i$ stands for fluid particle, and $I$ stands for interface node, which obtain:

$$ f^{s2f} = \sum_i repulse\_force(<i,I>) $$
$$ f^{f2s} = \sum_I -repulse\_force(<i,I>) $$

\section{weak form of structural governing equation}

\begin{enumerate}

\item{kinematic equation}

An accurate 2D kinemtic relationship is, details in \cite[p.~215]{limingrui}

\begin{subequations}
 \begin{align}
	U &= u - z \sin \theta \\
	W &= w - y(1- \cos \theta)
  \end{align}
 \end{subequations}

here, $u, w$ are displacements of neutral axis, and $\theta$ is rotational angle of an arbitrary cross-section.

\item{displacemnt-strain equation}

taking derivativs of kinematic equation, in terms of $x,y$ respectively:

 \begin{subequations}
	\begin{align}
	U_{,x} = u_{,x} - y \theta_{,x} \cos \theta \\
	U_{,y} = - \sin \theta \\
	V_{,x} = v_{,x} - y \theta_{,x} \sin \theta \\
	V_{,y} = -(1- \cos \theta)
	\end{align}
\end{subequations}

here we still adopt Euler-Boulluni Assumption, that the cross-section plane keep straight and vertical to neutral axis before and after deformation.

Giving Green-Lagrangian strain:
 \begin{subequations}
	\begin{align}
	E_{xx} = U_{,x} + ( U_{,x}^2 + V_{,x}^2)/2 = u_{,x} - y \theta_{,x} \cos \theta = u_{,x} + ( u_{,x}^2 + v_{,x}^2)/2 + y^2 \theta_{,x}^2/2 - y \theta_{,x}  \\
	E_{xy} = U_{,x} + V_{,x} + U_{,x} U_{,y} + V_{,x} V_{,y} = 0 \\ 
	E_{yy} = V_{,y} + ( U_{,y}^2 + V_{,y}^2)/2 = 0
	\end{align}
\end{subequations}

\item{virtual work and balance equation \_  weak form}

\subitem { virtual work equation}
In finite deformation theory, Green-Lagrangian Strain $E_{xx}$ is energy-conjugate couple correspond to Second Piola-Kirchhoff stress $S_{xx}$. so the internal force virtual work is

$$ \delta W^{in} = \int (S_{xx} \delta E_{xx}^T) dx =  \int (E_{xx} E \delta E_{xx}^T) dx $$

where $E$ is the linear elastic constant constitution module. 

$$ \delta W^{out} = \int q \delta w dx $$

\subitem{ balance equation and weak form}
 
Taken an infinite element $dx$,  consider vertical diretion deflection and moment equation,respectively.  we will obtian the balance equation as :

$$ q(x,t)dx + \frac{\partial Q}{\partial x} = \rho \frac{\partial ^2w}{\partial t^2} $$
 
where $q(x,t)$ is the transverse distribute force, $Q(x,t)$ is the internal  shear force.

$$  Q(x,t) = \frac{\partial M}{\partial x} + N \frac{\partial w}{\partial x} $$

substituting this equation into the above one, we obtain

$$ \rho \frac{\partial^2 w}{\partial t^2} + \frac{\partial^2}{\partial x^2}( EI \frac{\partial^2 w}{\partial x^2}) - N(x,t) \frac{\partial^2 w}{\partial x^2} = q(x,t) $$

where $N(x,t)$ is the normal force, and $M(x,t)$ is the moment. 

$$ N(x,t) = EA \epsilon(x,t) $$

Consider Green-Lagrangian Strain as before, and multiply $\delta v$ at both side of balance equation,and integration on the whole beam, we obtain the weak form:

$$ \int ( EI \frac{d^2 \delta v}{dx^2} \frac{d^2 w}{dx^2} + \rho \ddot{w} \delta v + N^L \frac{d w}{dx} \frac{d \delta v}{dx}) dx = \int q \delta v dx - v(0)Q(0) + \frac{dv}{dx}|_{x=0} M(0) + v(L)Q(L) - \frac{dv}{dx}|_{x=L}M(L) + N^L \frac{dw}{dx}v|^L_0 $$

consider boundary condition,
$ \delta v(0) = \frac{\delta v}{dx} = M(L) = Q(L) = \frac{\partial w}{\partial x}|_{x=0} = 0 $, so the weak form above is:

$$ \int ( EI \frac{d^2 \delta v}{dx^2} \frac{d^2 w}{dx^2} + \rho \ddot{w} \delta v + N^L \frac{d w}{dx} \frac{d \delta v}{dx}) dx = \int q \delta v dx + N^L \frac{dw}{dx}v _ {x=L} $$


\item { nodal force}

Beam nodes play dual-roles as FEM node as well as interface node, fluid-beam interaction is coupled by interface repulsive force $ f^{f2s}$, and this force is only acted at node point, which will lead singularity in the weak form of structural equation. so a simpler equivalent is:

$$  \int_0^L q(x,t) dx = \sum_{i=1}^{np} f^{f2s}_i $$

$$ \sum_{i=1}^{np-1} \int_{x_{i-1}}^{x_i} q_i dx = f_1 + \sum_{i=2}^{np-1} f_i + f_{np}$$ 

where $ q(x,t)$ is the equivalent distributed traction, and $np$ is the number of interface nodes. $q_i$ stands for a constant traction on an element length. 

so $q_i = \frac{f_i + f_{i+1}}{2l_e}, 1< i < np$,  $q_1 = (f_1 + f_2/2)/ l_e$, $q_{np} = (f_{np-1}/2 + f_{np})/l_e $

\item{matrix approximation}



 
\end{enumerate}

\section{SPH fluid equation}

assuming weak-compressible, the mass conservation equation is

$$ \frac{d \rho}{dt} = - \rho \nabla  v $$

momentum equation is 

$$ \rho \frac{dv}{dt} = - \nabla P + \mu \Delta^2 v + g $$

To update pressure here, introduce $p-\rho$ relationship

$$ p = c^2 \rho$$

where $ c^2 = 100 u^2 $ is virtual sound speed. 


\section{system approximation}

for sph particle velocity vector:

$$ \frac{du}{dt} = \sum_j m_i( \frac{\sigma_i}{\rho ^2 _i } + \frac{\sigma_j}{\rho^2 _j} )\nabla w_{ij} + \sum_j \frac{m_j}{\rho _j} f^{s2f}_j w_{ij} $$

Note to update the approximation fluid equation, we need calculate $\rho_i$, $w_{ij}$ at first.


for beam deflection displacement:

as before, we adopt Hermite shape function $ w = \sum_{k=1}^4 \Delta_k \theta_k $, then

$$ \sum_{J=1}^4 EI(\theta_I, \theta_J) \Delta_J - \rho \ddot{\Delta_J} l(\theta_I, \theta_J) = f_I^{f2s} $$


say $ \sum_{J=1}^4 EI(\theta_I,\theta_J) = K_{IJ}$, $ \rho l(\theta_I, \theta_J) = M_{IJ} $, $\mu \Delta u = L_i$, we can integratthe system equation in matrix form:


 $$ \begin{bmatrix} M && 0 \\ 0 && I \end{bmatrix} \{ \begin{array}{c} \ddot{\Delta_J} \\ \dot{v} \end{array} \} + \begin{bmatrix} K && 0 \\ 0 && L \end{bmatrix} \{ \begin{array}{c} \Delta_J \\ v \end{array} \} = \{ \begin{array}{c} f^{f2s}\\ f^{s2f} \end{array} \} $$

in which, $\Delta_J$ is the 4 DOF of FEM node, $v$ is SPH velocity vector(2 dimension), $L$ is Laplace Operator.



\section{Reference}
% \bibliography


\end{document}
