%========================================================================
\documentclass[11pt]{article}
\usepackage{graphics}
\graphicspath{{figs/}}

%===========================================
% Fix the margins
\setlength{\topmargin}{-0.4in}    % distance from top of page to begining of text
\setlength{\textheight}{9.0in}    % height of main text
\setlength{\textwidth}{6.5in}     % width of text
\setlength{\oddsidemargin}{0.in}  % odd page left margin
\setlength{\evensidemargin}{0.in} % even page left margin

%===========================================
\usepackage{amssymb,amsfonts}
\usepackage{amsmath}

%===========================================
\begin{document}

\title{SPH and Beam Equations}

\author{Zhengjiang Li}

\date{}

\maketitle

\section {Flow Governing Equations}

Starting from three conservational equations(mass, momentum and energy), here I most talk about the first two equations, which can be found in " the basic equations for fluid mechanics".

As standard SPH is assuming weakly-compressible. 

\begin{equation} \label{eq:mass}
	\frac{d \rho}{dt} = - \rho \nabla \cdot v
\end{equation}	

mass conservation in Lagrangain Description. $ \frac{d\rho }{dt}$ is material derivative.\ref{eq:mass}

\begin{equation} \label{eq:momentom}
	\rho \frac{dv}{dt} = \nabla \cdot \sigma + g
\end{equation}

momentum conservation in Lagrangian Description. $\sigma$ is stress, and it can be expressed as following,

$$ \sigma = -P + \lambda \nabla \cdot v + \mu (\frac{\partial u_i}{\partial x_j} +\frac{\partial u_j}{\partial x_i}) $$

\ref{eq:momentom}

Decompose material derivative of  LHS in the above equation, we will get the std NS equation.

As particles are moving but not fixed in the space, Lagrangian Description is suitable.

\section{SPH Approximations}

the basic idea behind SPH is  $ F(x) = \int F(x') \delta(x-x') dx' $, $\delta $is function defined only on a point,so SPH introduces the kernel function $w(x-x',h)$ to approximate $\delta$ in a sharp window $ (x-x_0, h)$

To make sure good approximation, kernel function should satisfy:

\begin{enumerate}
\item {regularization}

 $$ \int_{\Omega} W(x-x_j,h) dx_j = 1 $$

\item {compact support}

  $$  W(x-x_j) = 0  \forall x \in |x-x_j| > h $$

\item{convergence}
  as $$ \lim_{h \to 0}, W \to \delta $$
  
\end{enumerate}

there are many choices of kernel function, introductions can be found from open source document DualPhysics,SPHysics, GPUSPH and commercial code SPH FLOW

The role of kernel function is same as shape function in FEM, which is used to approxmite the variabls we are interested.  

After this process,another approximation is integration to finite summation. 
  $$ \int_{\Omega} f(x) W(x-x_j,h) dx_j = \sum_{i=1}^{N} \frac{m_i}{\rho_i} f(x_i) W(x-x_i,h) $$

For $ \rho \nabla \cdot f = \nabla \cdot (\rho f) - f \cdot \nabla \rho $

we can obtain

$$ \rho_i \nabla \cdot f(x_i) = \sum_j m_j[f(x_i) - f(x_j)] \nabla w_{ij} $$

For $ \frac{\nabla \cdot f}{\rho} = \frac{f}{\rho ^2} \nabla \rho + \nabla \frac{f}{\rho} $

similar, we can obtain

$$ \frac{\nabla \cdot f}{\rho} = \sum_j m_j (\frac{f_i}{\rho_i^2} + \frac{f_j}{\rho_j^2}) \nabla {w_{ij}} $$

\begin{enumerate}
	\item {density approximation}

	$$ \rho_i = \sum_{j=1}^{N}m_i W(r_{ij}) $$

	\item {accerlation approximation}
	
	$$ \frac{du}{dt} = \sum_j m_j( \frac{\sigma_i}{\rho_i^2} + \frac{\sigma_j}{\rho_j^2}) \nabla w_{ij} + \sum_j \frac{m_j}{\rho_j} f_j w_{ij} $$
	
\end{enumerate}

after that we can update velocity, position of particles.

\section{state equation}
To obtain the pressure field, as we assume weakly compressble, again SPH introduce an $\rho - p$  state-equation 

 $$ P = P_0 ((\frac{\rho}{\rho_0})^\gamma - 1) $$

in which, $ P_0$, $\rho_0$ are initial pressure and initial density. $\gamma$ usually give a value: 7.15

Another simple state equation we can use $p = c^2 \rho $.
in which $c$ is a kind of virtual sound speed, to get convergence, we need to make sure $ w = u^2 / c^2 = 0.01 $ 



\end{document}
