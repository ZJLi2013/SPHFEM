%========================================================================
\documentclass[11pt]{article}
\usepackage{graphics}
\graphicspath{{figs/}}

%===========================================
% Fix the margins
\setlength{\topmargin}{-0.4in}    % distance from top of page to begining of text
\setlength{\textheight}{9.0in}    % height of main text
\setlength{\textwidth}{6.5in}     % width of text
\setlength{\oddsidemargin}{0.in}  % odd page left margin
\setlength{\evensidemargin}{0.in} % even page left margin

%===========================================
\usepackage{amssymb,amsfonts}
\usepackage{amsmath}

%===========================================
\begin{document}

\title{Nonlinear Beam Element}

\author{Zhengjiang Li}

\date{}

\maketitle

\section{Beam Kinematics Assumption}

classic beam kinematics assumptions include, Euler-Bernoulli beam(EB) theory that neglects transverse shear strain; Timoshenko beam(TB) theory, that accounts for the transverse shear strain in a simpler way; and higher order beam theroy with additional terms into assumed displacement field.

in TB, the assumed displacement field is

$$ u(x,z) = u_0(x) + z \phi_x(x) $$
$$ v(x,z) = 0 $$
$$ w(x,z) = w_0(x) $$ 

in which $u, v, w$ are displacements in longitudinal, lateral and transverse respectively. $u_0, w_0$ denote displacement of a point on mid-plane of an undeformed beam along axial(x) and transverse(z) directions


in EB, the assumed displacement field is

$$ u(x,z) = u_0(x) - z \frac{dw_0}{dx} $$
$$ v(x,z) = 0 $$
$$ w(x,z) = w_0(x) $$

which means that the plane sections perpendicular to the mid-plane of the beam before deformation remain plane, and rotate such that they remain perpendicular to hte mid-plane after deformation.

\section{strain-displacement relations}

$$ \epsilon_{ij} = \frac{1}{2} ( \frac{\partial u_i}{\partial x_j} + \frac{\partial u_j}{\partial x_i}) + \frac{1}{2} ((\frac{\partial u_m}{\partial x_i} \frac{\partial u_m}{\partial x_j}) $$


according to Kirchhoff's hypothesis(plane strain), $\epsilon_{zz}$, $\epsilon_{xz}$, $\epsilon_{yz}$ equal zero, and for a thin beam(ration of length and radius is larger than 10), $\epsilon_{yy}$, $\epsilon_{xy}$ is zero. so the only nonzero strain is axial strain:

$$ \epsilon_{xx} = \frac{\partial u}{\partial x} + \frac{1}{2}[(\frac{\partial u}{\partial x})^2 + (\frac{\partial v}{\partial x})^2 + (\frac{\partial w}{\partial x})^2 $$

which gives,

$$ \epsilon_{xx} = \epsilon^0_{xx} + z \epsilon^1_{xx} $$
$$ \epsilon^0_{xx} = \frac{du_0}{dx} + \frac{1}{2} (\frac{dw_0}{dx})^2 $$
$$ \epsilon^1_{xx} = - \frac{d^2w_0}{dx^2} $$

From here, we can see there are 3 DOFs at each node $u, w, theta$. for EB theory, $ theta = \frac{\partial w}{\partial x} $, so actually we need only 2 DOFs at each node.
for Timonsenko theory, $u, w, theta$ are independent, os we need three DOFs at each node.

consider linear elastic of isotropic materials, $\sigma_{xx} = E \epsilon_{xx}$

\section{weak form}

consider virtual work principle and D'Alembert principle for dynamic problem, and only distributed pressure external force. we have:

$$ \delta W_{ext} = \delta W_{int} $$
$$ \delta W_{ext} = \int_0^L q\delta w dx$$
$$ \delta W_{int} = \sigma_{xx} \delta \epsilon_{xx} + \rho A \ddot{w} \delta w $$

for EB theory, $\delta \theta = - \frac{\delta w}{\delta x} $
for Timonsenko theory, $\delta \theta$ and $\delta w$ are independent.


$$ 	 \sigma_{xx} \delta \epsilon_{xx} = A \int ( \delta \epsilon^0_{xx} + z \delta \epsilon^1_{xx}) E_{11} (\epsilon^0_{xx} + z \epsilon^1_{xx}) dx  \\
   = A \int^L_0 ( E_{11} (\delta \epsilon^0_{xx} \epsilon^0_{xx} + z(\delta \epsilon^0_{xx} \epsilon^1_{xx} + \delta \epsilon^1_{xx} \epsilon^0_{xx}) + z^2(\delta \epsilon^1 \epsilon^1)]dx 
$$

define $C_1, C_2, C_3$ as 
$$ ( C_1, C_2, C_3) = \int E (1, z, z^2) dA $$


for $ \delta \epsilon^0_{xx} = \frac{d \delta u_0}{dx} + \frac{dw_0}{dx} \frac{d delta w_0}{dx} $, and $ \delta \epsilon^1_{xx} = - \frac{d^2 \delta \epsilon_0}{d x^2} $

\begin{multline}
 \delta w_{int} = \int_0^L ( C_1 [ \frac{d \delta u_0}{dx} ( \frac{du_0}{dx} + \frac{1}{2} (\frac{dw_0}{dx})^2) + \frac{d \delta w_0}{dx} \frac{dw_0}{dx} (\frac{du_0}{dx} + \frac{1}{2}(\frac{dw_0}{dx})^2)] \\
  - C_2 [\frac{d \delta u_0}{dx} (\frac{d^2w_0}{dx^2}) + \frac{d \delta w_0}{dx} \frac{d \delta w_0}{dx}  \frac{dw_0}{dx}(\frac{dw_0}{dx}{d^2w_0}{dx^2}) + \frac{d^2 \delta w_0}{dx^2} ( \frac{du_0}{dx} + \frac{1}{2} (\frac{dw_0}{dx})^2)] \\
  + C_3 \frac{d^2 \delta w_0}{dx^2} (\frac{d^2w_0}{dx^2})) +  \rho A \ddot{w} \delta w 
\end{multline}

\section{Finite Element Formulation}

for EB theory, approximation of u, w can be expressed as:

$$ u(x,t) = u_1(t) \psi_1(x) + u_2(t) \psi_2(x) $$
$$ w(x,t) = w_1(t) \phi_1(x) + theta_1(t) \phi_2(x) + w_2(t) \phi_3(x) + theta_2(t) \phi_4(x) $$

in which  $ \phi _i(t) = - \frac{d w_i(t)}{dx} $


\begin{multline}
\delta W_{in} = \int_0^L (C_1( \delta u_i \frac{d \psi_i}{dx}[\sum_{j=1}^2 u_j \frac{d \psi_i}{dx} + \frac{1}{2} \frac{dw}{dx} \sum_{J=1}^4 \Delta_J \frac{d \phi_J}{dx}] 
+ \delta \Delta_I \frac{d \phi_I }{dx} \frac{dw}{dx}[\sum_{j=1}^2 u_j \frac{d \psi_j}{dx} \frac{1}{2} \frac{dw}{dx} \sum_{J=1}^4 \Delta_J \frac{d \phi_J}{dx}])  \\
 - C_2( \delta u_i \frac{d \psi}{dx}(\sum_{J=1}^4 \Delta_J \frac{d^2 \phi_J}{dx^2}) + \delta \Delta_I \frac{d \phi_I}{dx} (\frac{w}{dx} \sum_{J=1}^4 \Delta_J \frac{d^2 \phi_J}{dx^2}) + \delta \Delta_I \frac{d^2 \phi_I}{dx^2} ( \sum_{j=1}^2 u_j \frac{d \psi_j}{dx} + \frac{1}{2} \frac{dw}{dx} \sum_{J=1}^4\Delta_J (\frac{d \phi_J}{dx}) \\
+ C_3( \delta \Delta_I \frac{d^2 \phi_I}{dx^2} \sum_{J=1}^4 \Delta_J \frac{d^2 \phi_J}{dx^2}) + \rho A \delta \Delta_i \phi_i \sum_{J=1}^4 \phi_J \frac{d^2 \Delta_J}{dt^2})dx
\end{multline}


%reference " linear and nonlinear FEM of advanced isotropical Beams

define $ \delta U^T = [ \delta u_1, \delta u_2, \delta w_1, \delta \theta_1, \delta w_2, \delta \theta_2]^T$; $ U = [u_1, u_2, w_1, \theta_1, w_2, \theta_2]; $

we can summary the above equation as:
$$ \delta W_{int} = \delta U^T [M] \{\ddot{U} \} + \delta U^T [K] \{U\} $$

similarly, external virtual work can be obtained as:

$$ \delta W_{ext} = \sum_{J=1}^4 \int_0^L q(x,t) \ddot{\Delta_i} \phi_i \phi_J \delta \Delta_J dx $$

in summary, we  obtain

$$ [M]\{\ddot{U}\} + [K]\{U\} = [F] $$





\end{document}
