%========================================================================
\documentclass[11pt]{article}
\usepackage{graphics}
\graphicspath{{figs/}}

%===========================================
% Fix the margins
\setlength{\topmargin}{-0.4in}    % distance from top of page to begining of text
\setlength{\textheight}{9.0in}    % height of main text
\setlength{\textwidth}{6.5in}     % width of text
\setlength{\oddsidemargin}{0.in}  % odd page left margin
\setlength{\evensidemargin}{0.in} % even page left margin

%===========================================
\usepackage{amssymb,amsfonts}
\usepackage{amsmath}

%===========================================
\begin{document}

\title{couple fluid and structural}

\author{Zhengjiang Li}

\date{}

\maketitle



\section {system governing equation}

In each time step, SPH for fluid, we calculate velocity of each particle, and then update the other properties of that point, such as its displacement,density and  pressure; and FEM for stuctural, we calculate beam deflection from the equation, and then update other properties, such as stress, strain.

so in this couple system, the self-dependent variable are particles' velocity vector $v(x,y,t)$ and FEM node position vector $w(x,t)$. the governing equation are:

   $$	\rho _i \dot{v_i} = \nabla \sigma _i + g_i + f_i^{s2f} $$ 
   $$	\frac{d^2}{dx^2}(EI \frac{d^2w_I}{dx^2} )- \rho_I \ddot{w_I} = f_I^{f2s} $$

the two equations are coupled by $f^{s2f}$ and $f^{f2s}$. as one FEM node correspond to multi fluid particles, while each fluid particles correspond to at most 2 FEM node, so these two forces are actually not equivalent.


Note in this simple case, all beam nodes play dual-roles as FEM node as well as interface nodes. 

one way is set interface node - fluid particle Pair $<i, I>$, here $i$ stands for fluid particle, and $I$ stands for interface node, which obtain:

$$ f^{s2f} = \sum_i repulse\_force(<i,I>) $$
$$ f^{f2s} = \sum_I -repulse\_force(<i,I>) $$

\section{approximation of equations}

for sph particle velocity vector:

$$ \frac{du}{dt} = \sum_j m_i( \frac{\sigma_i}{\rho ^2 _i } + \frac{\sigma_j}{\rho^2 _j} )\nabla w_{ij} + \sum_j \frac{m_j}{\rho _j} f^{s2f}_j w_{ij} $$

Note to update the approximation fluid equation, we need calculate $\rho_i$, $w_{ij}$ at first.


for beam deflection displacement:

as before, we adopt Hermite shape function $ w = \sum_{k=1}^4 \Delta_k \theta_k $, then

$$ \sum_{J=1}^4 EI(\theta_I, \theta_J) \Delta_J - \rho \ddot{\Delta_J} l(\theta_I, \theta_J) = f_I^{f2s} $$


say $ \sum_{J=1}^4 EI(\theta_I,\theta_J) = K_{IJ}$, $ \rho l(\theta_I, \theta_J) = M_{IJ} $, $\mu \Delta u = L_i$, we can integratthe system equation in matrix form:


 $$ \begin{bmatrix} M && 0 \\ 0 && I \end{bmatrix} \{ \begin{array}{c} \ddot{\Delta_J} \\ \dot{v} \end{array} \} + \begin{bmatrix} K && 0 \\ 0 && L \end{bmatrix} \{ \begin{array}{c} \Delta_J \\ v \end{array} \} = \{ \begin{array}{c} f^{f2s}\\ f^{s2f} \end{array} \} $$

in which, $\Delta_J$ is the 4 DOF of FEM node, $v$ is SPH velocity vector(2 dimension), $L$ is Laplace Operator.

the other parts of sph , FEM algorithm, please check $sph.pdf$ and $beam2.pdf$



\end{document}
