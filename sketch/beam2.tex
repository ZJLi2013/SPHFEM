%========================================================================
\documentclass[11pt]{article}
\usepackage{graphics}
\graphicspath{{figs/}}

%===========================================
% Fix the margins
\setlength{\topmargin}{-0.4in}    % distance from top of page to begining of text
\setlength{\textheight}{9.0in}    % height of main text
\setlength{\textwidth}{6.5in}     % width of text
\setlength{\oddsidemargin}{0.in}  % odd page left margin
\setlength{\evensidemargin}{0.in} % even page left margin

%===========================================
\usepackage{amssymb,amsfonts}
\usepackage{amsmath}

%===========================================
\begin{document}

\title{Nonlinear Beam Element}

\author{Zhengjiang Li}

\date{}

\maketitle

\section{Beam Kinematics Assumption}

classic beam kinematics assumptions include, Euler-Bernoulli beam(EB) theory that neglects transverse shear strain; Timoshenko beam(TB) theory, that accounts for the transverse shear strain in a simpler way; and higher order beam theroy with additional terms into assumed displacement field.

in TB, the assumed displacement field is

$$ u(x,z) = u_0(x) + z \phi_x(x) $$
$$ v(x,z) = 0 $$
$$ w(x,z) = w_0(x) $$ 

in which $u, v, w$ are displacements in longitudinal, lateral and transverse respectively. $u_0, w_0$ denote displacement of a point on mid-plane of an undeformed beam along axial(x) and transverse(z) directions


in EB, the assumed displacement field is

$$ u(x,z) = u_0(x) - z \frac{dw_0}{dx} $$
$$ v(x,z) = 0 $$
$$ w(x,z) = w_0(x) $$

which means that the plane sections perpendicular to the mid-plane of the beam before deformation remain plane, and rotate such that they remain perpendicular to hte mid-plane after deformation.

in our implementation, actually, we use much more simpler beam model, ignoring the axial deformation. so the kinematics equation is

$$ u(x,z) = -z \frac{dw} {dx} $$
$$ v(x,z) = 0 $$
$$ w(x,z) = w(x) $$


\section{strain-displacement relations}

$$ \epsilon_{ij} = \frac{1}{2} ( \frac{\partial u_i}{\partial x_j} + \frac{\partial u_j}{\partial x_i}) + \frac{1}{2} ((\frac{\partial u_m}{\partial x_i} \frac{\partial u_m}{\partial x_j}) $$


according to Kirchhoff's hypothesis(plane strain), $\epsilon_{zz}$, $\epsilon_{xz}$, $\epsilon_{yz}$ equal zero, and for a thin beam(ration of length and radius is larger than 10), $\epsilon_{yy}$, $\epsilon_{xy}$ is zero. so the only nonzero strain is axial strain:

$$ \epsilon_{xx} = \frac{\partial u}{\partial x} + \frac{1}{2}[(\frac{\partial u}{\partial x})^2 + (\frac{\partial v}{\partial x})^2 + (\frac{\partial w}{\partial x})^2 $$

which gives,

$$ \epsilon_{xx} = -z \frac{d^2 w}{dx^2} + \frac{1}{2} (\frac{dw}{dx})^2 $$
in general, 2D beam element will have 3 Degree of Freedom(DoF)s: $u, w, \theta$. in Timonsenko Theory, there are 3 DOFs; As the plane undeformed assumption in Euler-Bounulli Theory, there are only $u, w$ in EB beam. For our implementation, we use only $w$ as we said in kinemtic section.

consider linear elastic of isotropic materials, $\sigma_{xx} = E \epsilon_{xx}$

\section{weak form}

 consider balance equation, and ignore body force, we obtain

 $$ \frac{d^2}{dx^2}(EI \frac{d^2w}{dx^2}) + b = q $$

in which , the body force in unit volume/length  $b$ is  inertia force 
$$ b = - \rho \ddot{w} $$

$q$ is the distributed tractional force from external(fluid environment).

multiplying $\delta v$ and integrating over the interval $[0,L]$, we obtain:

$$ \int_0^L (EI \frac{d^2 \delta v}{dx^2} \frac{d^2w}{dx^2} + b \delta v)dx = \int_0^L v q dx + v(0) Q_1 + (-\frac{dv}{dx})_{|x=0} Q_2 + v(L) Q_3 + (-\frac{dv}{dx}_{|x=L} Q_4 $$

where $[Q_1, Q_2, Q_3, Q_4] = [-V(0), -M(0), V(L), M(L)] $, $V$ is shear force vertical to cross-section, $M$ is moment on cross-section.

\section{finite element formulation}

here we need Hermite shape function

$$ w(x,t) = w_1(t) \phi_1(x) + \theta_1(t) \phi_2(x) + w_2(t) \phi_3(x) + \theta_2(t) \phi_4(x) $$

define master element on length $l_e$ with coordinate $\xi$, ranging from $[-1, 1]$

$$ \phi_1(x) = \frac{1}{4} (1-\xi)^2 (2+\xi) $$
$$ \phi_2(x) = \frac{l_e}{8}(1-\xi)^2(1+\xi) $$
$$ \phi_3(x) = \frac{1}{4} (1+\xi)^2 (2-\xi) $$
$$ \phi_4(x) = \frac{l_e}{8}(1+\xi)^2(\xi-1) $$

define $[w_1, \theta_1, w_2, \theta_2] = [ \Delta_1, \Delta_2,\Delta_3, \Delta_4] $

$$ \sum_{j=1}^4 ( \int_0^L( EI \frac{d^2 \phi_i}{dx^2} \frac{d^2 \phi_j}{dx^2}\Delta_j - \rho \frac{d^2 \Delta_j}{dt^2} \phi_i \phi_j) dx) = \int_0^L \phi_i q(x,t) dx - Q_i $$  
% this is actually on an element


%$$ \sum_{j=1}^4 M_{ij} \ddot{\Delta_j} + \sum_{j=1}^4 K_{ij} \Delta_j = F_i $$ 

\section{ nodal force}

replace distribute force $\int_0^L \phi_i q dx $ with nodal force, cause we can easily obtain nodal force in this coupled design.

so in every element, we have:

$$ \sum_{j=1}^4 ( \int_0^L( EI \frac{d^2 \phi_i}{dx^2} \frac{d^2 \phi_j}{dx^2}\Delta_j - \rho \frac{d^2 \Delta_j}{dt^2} \phi_i \phi_j) dx) = F_i - Q_i $$  

$Q_i$ is nonzero at the support end, the other end is free but will have nodal force from fluid.  

this nodal force is obtained from the neighboring searching domain of each interface node.






\end{document}
